\documentclass[fleqn,12pt,openany]{book}

% These two need to be set before including Cleaver style package
\title{Cleaver User Guide}
\author{Jonathan R. Bronson}

% INCLUDE Cleaver STYLE DOCUMENT
\usepackage{fancyvrb}
\usepackage{cleaver}
\usepackage{minted}


\begin{document}

%% 


% CREATE TITLE PAGE --------------------------------------------------
\maketitle

\chapter{Cleaver Overview}

Cleaver is a C++ library for generating guaranteed-quality, tetrahedral 
meshes of multimaterial volumetric data. It is based on the 'Lattice Cleaving' 
algorithm:\\
\\
Bronson J., Levine, J., Whitaker R., "Lattice Cleaving: Conforming Tetrahedral
Meshes of Multimaterial Domains with Bounded Quality". Proceedings of the 21st
International Meshing Roundtable (San Jose, CA, Oct 7-10, 2012) \\
\\
The method is theoretically guaranteed to produce valid meshes with bounded 
dihedral angles, while still conforming to multimaterial material surfaces. 
Empirically, these bounds have been shown to be significant.\\
\\
In addition to the C++ library, the source includes a command-line
application for generating meshes. It makes use of the 
\href{http://teem.sourceforge.net/nrrd/index.html}{Teem NRRD Library}
for loading input volumes.

\chapter{Building Cleaver from Source}

\section{Prerequisites}

\begin{itemize}
\item CMake (can create GNU make, Visual Studio, NMake, and XCode projects)
\item C/C++ compiler
\item GNU make on Linux/Unix platforms
\end{itemize}

The free Visual Studio Express compiler is available for Windows.

\section{Download Sources}

Download the source archive from
\href{http://software.sci.utah.edu}{SCI software portal} from the Cleaver software page.
These are updated regularly, and when a major bug fix has been applied. Only
gzipped tar archives are available at the moment. The \href{http://www.gzip.org/}{gzip} 
product page also has a list of tools that can unpack the archive.

\section{Linux and OS X}

Use the CMake tool \textbf{ccmake} to configure the Cleaver build on the
command line. The command-line tool is disabled by default because it
relies on a 3rd party Teem package, which may not compile on some systems.
Alternatively use \textbf{cmake} to generate a default build:

\begin{Verbatim}[frame=single]
cmake <src>
make
\end{Verbatim}

\section{Windows}

CMake can generate Visual Studio project files. However, we highly recommend 
using \href{http://qt-project.org/wiki/Category:Tools::QtCreator}{Qt Creator}, 
a cross-platform IDE with built-in support for loading CMake projects
directly.

Alternatively, CMake can generate NMake files for command line compiling:

\begin{Verbatim}[frame=single]
cmake <src>
nmake
\end{Verbatim}



\chapter{Datatypes}

All classes in the Cleaver library are defined under
the \emph{Cleaver} namespace. Types and functions 
can be accessed using the prefix Cleaver:: or by
'using namespace Cleaver;' The details of each type
can be found in their corresponding header files
and are fairly self documenting. What follows is
a more detailed explanation of the types and how
they are intended to be used.


\section{Utility Types}

\subsubsection{vec3}
The core datatype used throughout Cleaver is 3d vector
class, \emph{vec3}. It is the only lower-case class in
the Library, and breaks naming convention to indicate
that it is meant to be used as a lightweight primitive.
Both scalar and vector operators are defined for the class,
along with some other convenience functions.

\subsubsection{BoundingBox}
\emph{BoundingBox} is provided as a convenience class for
representing field and volume bounds. It has a vec3 \emph{origin}
and a vec3 \emph{size}, along with several convenience functions.


\section{Input Types}

The lattice cleaving algorithm takes as input a set of n signed-indicator 
functions, corresponding to n input materials. These functions represent a 
measure of the 'presence' or 'strength' of a material at any given point in 
space. These indicator functions can be for instance, partial mixture models,
or signed distance fields, it does not matter. The material at any given
point is interpreted as the material whos function is maximum at that point.
Ties indicate a material interface.

\subsubsection{ScalarField}
Cleaver makes use of inheritance and polymorphism for input data. \\
The base class for all indicator functions is the \emph{ScalarField} class.
It is a virtual base class, and is used only as an interface. Any class
that implements this interface must define two methods: \emph{valueAt()}
and \emph{bounds()}.

\subsubsection{FloatField}
\emph{FloatField} is an implementation of the \emph{ScalarField} interface.
It is a wrapper for loading 3d arrays of floating point data. 

\subsubsection{InverseField}
\emph{InverseField} is a wrapper class that inverts the sign of another
\emph{ScalarField}. Because the lattice cleaving algorithm requires at
least two input fields, this is a convenient class when only one input
field is given. Example:

\begin{minted}[mathescape,
    linenos,
    numbersep=5pt,
    gobble=2,
    frame=lines,
    framesep=2mm]{cpp}
    
    Cleaver::ScalarField *field1 = loadNRRDFile(filename);
    Cleaver::ScalarField *field2 = new Cleaver::InverseField(field1);
\end{minted}


\subsubsection{Volume}

The \emph{Volume} class encapsulates an entire input to the mesher. It
contains a set of indicator functions (\emph{i.e. ScalarFields}) and
a desired output size. A Volume object can be constructed by passing
in a std::vector of ScalarFields, and optionally an output size. If
no size is specified, the output size will equal the size of the first
input field.



\subsubsection{PaddedVolume}

One of the current limitations of Cleaver is it cannot handle material
interfaces on the boundary of the domain. Until this issue is addressed,
the \emph{PaddedVolume} class is a convenient solution. The constructor
takes in an existing Volume and optionally the padding thickness and 
padding values. Example:

\begin{minted}[mathescape,
    linenos,
    numbersep=5pt,
    gobble=2,
    frame=lines,
    framesep=2mm]{cpp}
    
    std::vector<Cleaver::ScalarField*> fields = loadNRRDFiles(inputs);
    if(fields.empty()){
        std::cerr << "error loading input" << std::endl;
        return 0;
    }
    Cleaver::Volume *volume = new Cleaver::Volume(fields);
    Cleaver::PaddedVolume *pVolume = new Cleaver::PaddedVolume(volume);
\end{minted}

\subsection{Creating Custom Fields}

Custom indicator functions can be created by subclassing the ScalarField 
class. This is especially useful if the input can be described analytically, 
or is accessed through a 3rd party library.

\paragraph{}
For example, suppose we want to create an indicator function for a sphere.
Let's call the containing class \emph{SphereField}. We can define as following:

\begin{minted}[mathescape,
    linenos,
    numbersep=5pt,
    gobble=2,
    frame=lines,
    framesep=2mm]{cpp}
  #include <Cleaver/ScalarField.h>    

  class SphereField : public ScalarField
  {
    public:
    SphereField(const Cleaver::vec3 &center, float r,
                const Cleaver::BoundingBox &bounds);

    virtual float valueAt(float x, float y, float z) const;
    virtual float valueAt(const Cleaver::vec3 &x) const;
    virtual Cleaver::BoundingBox bounds() const;

    private:
    Cleaver::BoundingBox m_bounds;
    Cleaver::vec3        m_center;
    float                m_radius;
  };
\end{minted}

Note, here we overrided both valueAt() methods, though only the one taking
individual x,y,z coordinates is required in order to subclass ScalarField.
Overloading the valueAt() operator as well is encouraged. In this example,
it allow us to write a cleaner method:

\begin{minted}[mathescape,
    linenos,
    numbersep=5pt,
    gobble=2,
    frame=lines,
    framesep=2mm]{cpp}

  float SphereField::valueAt(const Cleaver::vec3 &x);
  {
    return = m_radius - std::abs(length(x - m_center));
  }
\end{minted}

\newpage

\section{Output Types}

\subsubsection{TetMesh}

The \emph{TetMesh} class is the sole output of Cleaver. On 
construction it is composed of a list of \emph{Vertex3D} 
pointers, and a list of \emph{Tet} pointers. Methods are 
available for writing the TetMesh directly to several file
formats, including \emph{TetGen}, \emph{SCIRun}, and \emph{MATLAB}.

\begin{minted}[mathescape,
    linenos,
    numbersep=5pt,
    gobble=2,
    frame=lines,
    framesep=2mm]{cpp}

    class TetMesh
    {    
      public:

      TetMesh(std::vector<Vertex3D*> &verts, std::vector<Tet*> &tets);    
      ~TetMesh();
      ...
      void writeMatlab(const std::string &filename, bool verbose = false);
      void writeNodeEle(const std::string &filename, bool verbose = false);
      void writePtsEle(const std::string &filename, bool verbose = false);
      void writePly(const std::string &filename, bool verbose = false);
      ...
      std::vector<Vertex3D*> &verts;
      std::vector<Tet*> &tets;
      Face* faces;
      int nFaces;
      ...
    };
\end{minted}

\subsubsection{Tet}
The \emph{Tet} class is composed of an array of pointers to the 4 vertices 
composing the tet. It also contains a material label \emph{mat\_label}.


\begin{minted}[mathescape,
    linenos,
    numbersep=5pt,
    gobble=2,
    frame=lines,
    framesep=2mm]{cpp}

    class Tet
    {
      public:
      ...
      Vertex3D *verts[4];
      char mat_label;
      ...
    };
\end{minted}

\subsubsection{Vertex3D}
The \emph{Vertex3D} class contains a vec3 of its coordinates, and
a std::vector of the Tet's containing it. The coordinates can be
accessed through the pos() method. 

\begin{minted}[mathescape,
    linenos,
    numbersep=5pt,
    gobble=2,
    frame=lines,
    framesep=2mm]{cpp}

  class Vertex3D : public Geometry{
    ...
    std::vector<Tet*> tets;
    vec3& pos();
  };

\end{minted}

\chapter{Using Cleaver}

To use the Cleaver library, include the header
$\langle$Cleaver/Cleaver.h$\rangle$. This gives 
access to the core classes and functions needed for 
meshing, including \emph{CleaverMesher}. 

\paragraph{}
The \emph{CleaverMesher} class can be used by passing in a Volume
containing at least two indicator functions (\emph{i.e. Scalar Fields}),
and then calling \emph{createTetMesh()}. Finally, calling \emph{getTetMesh()}
will return the TetMesh object. Setters and getters are available to change
the current volume, along with a \emph{cleanup()} method to clear out
temporary data for previous volumes.

\paragraph{}
The preferred way to mesh with cleaver is to use the static method:

\begin{minted}[mathescape,
    linenos,
    numbersep=5pt,
    gobble=2,
    frame=lines,
    framesep=2mm]{cpp}

  TetMesh* createMeshFromVolume(const Cleaver::AbstractVolume *volume);
\end{minted}

This function takes care of all construction and cleanup of the internals
of the mesher. The user is still responsible for freeing the Volume and
TetMesh from memory once they are no longer needed. Using the Teem NRRD 
library for input, an example of loading a dataset and meshing it with 
this function would be as follows:

\begin{minted}[mathescape,
    linenos,
    numbersep=5pt,
    gobble=2,
    frame=lines,
    framesep=2mm]{cpp}
    
    std::vector<Cleaver::ScalarField*> fields = loadNRRDFiles(inputs);
    if(fields.empty()){
        std::cerr << "error loading input" << std::endl;
        return 0;
    }
    Cleaver::Volume *volume = Cleaver::Volume(fields);
    Cleaver::TetMesh *mesh = Cleaver::createMeshFromVolume(volume);
\end{minted}

\chapter{Command-Line Interface}
\label{sec:commandline}

Some users may wish to use Cleaver in a non-programming environment.
The command line interface provides a convenient way to do this. It
takes \href{http://teem.sourceforge.net/nrrd/index.html}{Teem NRRD} 
files as input, and writes a mesh directly to the file system.

\section{Usage}

Running the command line program with the help flag -h produces the
usage guide:\\

\begin{Verbatim}[frame=single]

usage: cleaver -p -as [val] -al [val] -i [input1.nrrd input2.nrrd ...]
               -o [output]  -f [format]

         -p      use padding        default=no
         -as     alpha short        default=0.357
         -al     alpha long         default=0.203
         -i      input filenames    minimimum=1
         -o      output filename    default=output
         -f      output format      default=tetgen
         -h      help

   Valid Parameters:
                 alpha short        0.0 to 1.0
                 alpha long         0.0 to 1.0
                 tetmesh formats    tetgen, scirun, matlab
\end{Verbatim}

These input flags are currently order-dependent. 
\newpage

Typically use cases are:


\begin{Verbatim}[frame=single]
Examples:
cleaver -h                                       print help guide
cleaver -i mat1.nrrd mat2.nrrd                   basic use case
cleaver -p -i mat1.nrrd mat2.nrrd                pad boundary
cleaver -i mat1.nrrd mat2.nrrd -o mesh           specify output name
cleaver -i mat1.nrrd mat2.nrrd -f scirun         specify output format
\end{Verbatim}

\chapter{Known Issues}

Cleaver is still currently under heavy development. If a feature
does not work properly or does not exist, please inform us so that
we can work on it for a future release. In the meantime, here is a
list of issues we are aware of and working on:


\begin{itemize}
\item \textbf{Geometric Predicates}: \\
  Cleaver does not yet use robust geometric
  predicates. This can lead to problems for ill-conditioned inputs. If
  computations result in NaN values, Cleaver produces a warning and
  simplifies the calculation (e.g. using barycenters vs. computed points).
\item \textbf{Boundary Interfaces}:  \\
  Cleaver does not yet support material interfaces on the boundary of
  the volume domain. If they are encountered, the mesher will stop and
  warn the user to rerun with padding. Using the PaddedVolume class in
  the library, or calling the CLI with the padding flag -p is fast way
  around this issue.
\end{itemize}

\appendix
\chapter{License}

The Cleaver Library and accompanying source is released under 
the MIT license.\\
\\
Copyright (C) 2012, Jonathan Bronson,\\
 Scientific Computing \& Imaging Institute,\\
 University of Utah\\
\\
 Permission is  hereby  granted, free  of charge, to any person
 obtaining a copy of this software and associated documentation
 files (the "Software"), to deal in the Software without
 restriction, including  without limitation the rights to  use,
 copy, modify,  merge, publish, distribute, sublicense,  and/or
 sell copies of the Software, and to permit persons to whom the
 Software is  furnished  to do  so,  subject  to  the following
 conditions:\\
\\
 The above  copyright notice  and  this permission notice shall
 be included  in  all copies  or  substantial  portions  of the
 Software.\\
\\
 THE SOFTWARE IS  PROVIDED  "AS IS",  WITHOUT  WARRANTY  OF ANY
 KIND,  EXPRESS OR IMPLIED, INCLUDING  BUT NOT  LIMITED  TO THE
 WARRANTIES   OF  MERCHANTABILITY,  FITNESS  FOR  A  PARTICULAR
 PURPOSE AND NONINFRINGEMENT. IN NO EVENT  SHALL THE AUTHORS OR
 COPYRIGHT HOLDERS  BE  LIABLE FOR  ANY CLAIM, DAMAGES OR OTHER
 LIABILITY, WHETHER IN AN ACTION OF CONTRACT, TORT OR OTHERWISE,
 ARISING FROM, OUT OF OR IN CONNECTION WITH THE SOFTWARE OR THE
 USE OR OTHER DEALINGS IN THE SOFTWARE.



\end{document}
